\documentclass[12pt,letterpaper]{article}
\usepackage[utf8]{inputenc}
\usepackage[spanish]{babel}
\usepackage[T1]{fontenc}
\usepackage[ansinew]{inputenc}
\usepackage{graphicx}
%!!!!!!!!!!!!!!!!!!!!!!!!!!!!!!!!!!!!!!!!!!!!!!!!!!!!




% Enable SageTeX to run SageMath code right inside this LaTeX file.
% documentation: http://mirrors.ctan.org/macros/latex/contrib/sagetex/sagetexpackage.pdf
% \usepackage{sagetex}
\begin{document}
\begin{portada}\begin{center}
\vspace*{-1in}
\begin{figure}[htb]
\begin{center}
\includegraphics[width=4cm, height=4cm]{escudo.png}
\end{center}
\end{figure}

UNIVERSIDAD DE SONORA\\
\vspace*{0.15in}
DEPARTAMENTO DE CIENCIAS EXACTAS Y NATURALES \\
\vspace*{0.6in}

\begin{Large}
\textbf{CONOCIENDO NUESTRA ATMÓSFERA} \\
\end{Large}
\vspace*{0.3in}

\vspace*{0.3in}
\rule{80mm}{0.1mm}\\
\vspace*{0.1in}
\begin{large}
Por Alvarez Sanchez Franscisco Eduardo
\end{large}
\vspace*{0.1in}\\\\
\begin{large}
Maestro: Dr. Carlos Lizárrga Celaya \\
\end{large}
\vspace*{5cm}
\begin{large}
Hermosillo, Sonora a 30 de Enero de 2017
\end{large}
\end{center}

\end{titlepage}

%\includegraphics[width=4cm, height=4cm]{escudo.png}
%\title{Conociendo nuestra Atmósfera}
%\author{Álvarez Sánchez Francisco Eduardo\\\\
%Dr. Carlos Lizárraga Celaya}



La atmósfera la podemos definir como una mezcla de varios gases que rodean algún cuerpo celeste, como la tierra. Ya que la tierra posee este campo de gases, es un buen objeto de estudio.

En el siguiente documento nos enfocaremos explicar de que se comportan cada capa de la atmósfera y hablaremos de las propiedades de cada una.
\\Estudiaremos cual es la composición de cada una de las capas, que tipo de elementos la constituyen, entre otros puntos.
\\Se incluirán imágenes para tratar de describir comportamientos de los procesos que realiza la atmósfera.
\section{¿Qué es la atmósfera?}

La palabra atmósfera es un termino compuesto por dos partículas, átomos, que en griego significa vapor, aire y la palabra esfera. Es decir que es la envoltura gaseosa que cubre a una esfera o cuerpo celeste o a un planeta.

La atmósfera es una envoltura gaseosa que rodea la tierra esa envoltura esta constituida por el aire, que es una mezcla de gases y vapores conteniendo en suspensión materias sólidas, finamente divididas, así también iones y hasta partículas nucleares en sus regiones mas alejadas de la superficie terrestre
\\A través de sucesivas investigaciones y con el transcurso del tiempo, se ha dividido a la atmósfera por sus características en varias capas.

\section{Capas y sus características}
La atmósfera se encuentra constituida por cuatro capas principales conocidas como la tropósfera, estatósfera, mesósfera y termósfera; sin embargo en este documento tomaremos en cuenta una quinta conocida como la exósfera.
\begin{center}
\includegraphics[width=10cm, height=10cm]{atmosfera}
\end{center}

\subsection{Tropósfera}
Esta capa se le conoce como la base de la atmósfera, en ella se producen los fenómenos meteorológicos que conocemos que son: nubes, frentes, nieblas, bruma, tempestades de polvo ó arena,etc. Mas del setenta y cinco por ciento del peso total del aire, casi toda la humedad y la mayor parte del polvo atmósferico están contenidos en esta capa La densidad del aire disminuye con la altura, al igual que la presión, la temperatura y la humedad.

La tropósfera esta constituida por una mezcla de gases. Una muestra de aire puro y seco contiene 75 por ciento de Nitrógeno, 21 por ciento de Oxígeno, 0.9 por ciento de Argón y 0.03 por ciento de Bióxido de Carbono. Existe un 0.01 por ciento de gases varios, que son rastros de Neón, Criptón, Hélio, Ozóno, Xenón é Hidrógeno, siendo esta cantidad tan pequeña que estos gases no tienen mucha importancia práctica para el estudio de lo antedicho. 


\subsection {Estratósfera}
En esta capa hay escaso movimiento de las masas de aire que la forman. La temperatura permanece estacionaria en las capas inferiores, aumentando bruscamente en u limite superior. La humedad es tan escasa en esta región que muy raramente se producen nubes.


\subsection{Mesósfera}
La radiación solar disocia en esta capa las pocas moléculas de vapor de agua allí existentes. El oxígeno se transforma en ozono producto de equilibrios fotoquímicos. La presión se reduce a los 50 km, aproximadamente a la milésima parte de las registrada a nivel del mar. La temperatura desciende con la altura en esta capa. 


\subsection{Termósfera o ionósfera}
La termósfera es la capa en la que la temperatura aumenta con la altura hasta alcanzar 1500°C a los 300 km, aproximadamente. Existen ciertos niveles donde se acumulan partículas que se encuentran cargadas de electricidad. 

En esta capa podemos encontrar subcapas: Capa D y Capa E ó Capa de Kenelly Heavside.

En la parte inferior de la Termósfera se produce las llamadas nubes noctilucentes, que son masas de partículas finamente divididas en suspensión y que proceden de las erupciones volcánicas ó del espacio extraterrestre. También en esta capa se pueden producir las auroras polares. 

\subsection {Exósfera o magnetósfera}
A esta capa se le considera como el limite superior de la atmósfera, las partículas materiales están a tan íntimo número que pueden hacer largos recorridos sin chocarse unas con otras.

\section {Transferencia de calor}
Existen tres tipos de transmisión de calor: radiación que es la transimisión de energía directamente del sol a la tierra sin ayuda de ningún medio material, conducción que es el transporte de calor a través de la materia por contacto molecular y la advección que es la transferencia de calor que produce el aire en movimiento.

La radiación solar es causante de la energía que mantiene y produce todos los procesos atmósfericos sobre la tierra, jugando un papel muy importante la tierra en la conversión de esa radiación en calor sensible y en la distribución sobre si misma y sobre la atmósfera.
\begin{center}
\includegraphics[width=10cm, height=5cm]{calor}
\end{center}

En la conducción la energía calórica pasa de un a otra molécula dentro del cuerpo, por ejemplo el exceso de energía calórica en un extremo de una varilla se deslaza hacia el otro extremo, adquiriendo así una temperatura uniforme.

La transmisión se produce por advección cuando corrientes de aire o de agua mueven masas calentadas o enfriadas desde un lugar a otro, donde ceden parte de su calor almacenado. En los procesos meteorológicos la transmisión por convección se designa como el término advección, reservando el de convección para transmisión en sentido vertical.
\section {El água}
El água existe en la atmósfera en tres estados físicos: Sólido, líquido y gaseoso. Sólido en forma de cristales de hielo, líquido en las nubes y nieblas que se forman a partir de pequeñas gotas de agua y en la lluvia; y por último de forma gaseosa que se le conoce como vapor de agua que es considerado uno de los componentes más importantes de la atmósfera. 
\begin{center}
\includegraphics[width=10cm, height=5cm]{agua}
\end{center}
\section{¿De dónde viene esta información?}
Existen varias formas para adquirir información de lo que sucede en nuestra atmósfera, sin embargo el método más utilizado es el de los globos meteorológicos o globo sonda. Una de las primeras personas en utilizar estos globos fue León Teissenrenc de Bort, un meteorólogo francés.

Este es un globo aerostático que eleva instrumentos en la atmósfera para suministrar información acerca de la presión atmosférica, la temperatura y la humedad por medio de un pequeño aparato de medida desechable llamado radiosonda.Para obtener información del viento, los globos pueden ser rastreados por GPS.
\begin{center}
\includegraphics[width=10cm, height=5cm]{globo}
\end{center}

\section{Bibliografía}

- Ricardo Santiago Netto. (2011). Meteorología básica. 29-01-2017, de FisicaNet Sitio web: http://www.fisicanet.com.ar/monografias/monograficos2/es17_meteorologia.php\\\\


- Paul-Davies, Steven (2002). The Prisoner Handbook. London: Pan Books.\\\\

-Staff (Febrero 1958). "Chief Special Projects Section: Dr. Lester Machta" (PDF). United States Weather Bureau: 39–41. 29-01-2017.\\\\

-http://www.meted.ucar.edu/mesoprim/skewt/navmenu.php?tab=1&page=3.0.0&type=flash a 29-01-2017








\end{document}
