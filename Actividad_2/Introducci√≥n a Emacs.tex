\documentclass[12pt,letterpaper]{article}
\usepackage[utf8]{inputenc}
\usepackage[spanish]{babel}
\usepackage[T1]{fontenc}
\usepackage[ansinew]{inputenc}
\usepackage{graphicx}

\begin{document}
\begin{portada}\begin{center}
\vspace*{-1in}
\begin{figure}[htb]
\begin{center}
\includegraphics[width=4cm, height=4cm]{escudo.png}
\end{center}
\end{figure}

UNIVERSIDAD DE SONORA\\
\vspace*{0.15in}
DEPARTAMENTO DE CIENCIAS EXACTAS Y NATURALES \\
\vspace*{0.6in}

\begin{Large}
\textbf{Introducción a Emacs} \\
\end{Large}
\vspace*{0.3in}

\vspace*{0.3in}
\rule{80mm}{0.1mm}\\
\vspace*{0.1in}
\begin{large}
Por Álvarez Sánchez Francisco Eduardo
\end{large}

\begin{large}
Maestro: Dr. Carlos Lizárrga Celaya \\
\end{large}
\vspace*{5cm}
\begin{large}
Hermosillo, Sonora a 8 de Febrero de 2017
\end{large}
\end{center}
\thispagestyle{empty}





\\\\\\

\newpage

¿Cual es tu primera impresión del uso de bash/Emacs?
No tuvimos una primera impresión de Emacs ya que este ya lo habíamos utilizado anteriormente cuando programabamos en FORTRAN, sin embargo si conocimos algunas formas de utilizarlo diferentes a las que ya conocíamos y que probablemente sean muy útiles en un futuro. 


¿Ya lo habías utilizado? 
Si, como se comento en la preguna anterior, ya lo habíamos usado al programar en lenguaje fortran. 


¿Qué cosas se te dificultaron más en bash/Emacs?
Quiza no hubo nada que fuera tan complicado o algo que leyendo alguna guía no se pudiera conocer. 


¿Qué ventajas les ves a Emacs?
Como comento nuestro profesor, una de las principales ventajas de Emacs es que es uno de los mas utilizados, por lo tanto puede que sea mas fácil encontrar información de como trabajar con el. También cabe destacar que no es un procesador de texto muy complicado de utilizar.


¿Qué es lo que mas te llamó la atención en el desarrollo de esta actividad?
Me llamo mucho la atención la facilidad con la que alguien sin conocimientos previos de programación muy avanzada puede realizar tareas que pueden denominarse complicadas como obtener datos atmosféricos desde una pagina de internet dedicada a esta tarea.


¿Qué cambiarías en esta actividad?
No cambiaría nada, no por el momento ya que a mi me fue lo suficientemente útil y aprendi bastantes cosas.


¿Que consideras que falta en esta actividad? 
Desde mi punto de vista, no vi necesario incluir otra cosa.


¿Algún comentario adicional que desees compartir? 
Quizá para nosotros fue bastante sencillo ya que contabamos con un profesor que nos guiaba en todo momento, sin embargo esta tarea quizá no sea tan sencilla al momento de no tener alguna ayuda.


\end{document}