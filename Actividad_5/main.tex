\documentclass[12pt]{article}
\usepackage[utf8]{inputenc}
\usepackage[T1]{fontenc}
\usepackage[spanish]{babel}%caracteres en español
\usepackage{verbatim}
\title{\huge \textbf{\textsc{Mareas y corrientes}}}%titulo en grande-negritas-versalitas
\author{Álvarez Sánchez Francisco Eduardo}
\usepackage{graphicx}%para cargar imagenes
\usepackage{wrapfig} %para acomodar figuras y que compartan espacio con texto
\usepackage{fancyhdr}
\pagestyle{fancy}
\fancyhf{}
\usepackage{enumerate}
\usepackage{cite}
\usepackage{hyperref}
\usepackage{bookmark}
\fancyfoot[R]{Página \thepage}
\setlength\headheight{15 pt}
\fancyhead[L]{Francisco Eduardo Álvarez Sánchez}
\fancyhead[R]{Física Computacional I}
\usepackage{booktabs}
\usepackage[nottoc,numbib]{tocbibind}


\author{Álvarez Sanchez Francisco Eduardo}
\title{\textbf{\textsc{Mareas y Corrientes}}} 
\date{\today}

\begin{document}

\begin{titlepage}
	\centering
    \begin{figure}[ht!]
    \centering
    \includegraphics[scale=0.5]{escudo.png}
    
    \textbf{UNIVERSIDAD DE SONORA \\ DIVISIÓN DE CIENCIAS EXACTAS Y NATURALES \\ DEPARTAMENTO DE FÍSICA \\ LICENCIATURA EN FÍSICA}
	\maketitle
    \hrule \bigskip
    \large{Física computacional I}\\
	Profr. Carlos Lizárraga Celaya
    \end{figure}
\thispagestyle{empty}
\end{titlepage}

\newpage

\section{Resúmen}
\noindent Consideramos a las mareas como el cambio periódico del nivel del mar producido principalmente por las fuerzas de atracción gravitacional que ejercen el Sol y la Luna sobre la tierra. 

Otros fenómenos ocasionales, como los vientos, las lluvias, el desborde de ríos y los tsunamis provocan variaciones del nivel del mar, todos ocasionales, producen variaciones en el nivel del mar, sin embargo estas no son tomadas como mareas porque no son ocacionadas por la fuerza gravitacional ni tienen periodicidad. 

\section{Introducción}
 
\noindent En el siguiente trabajo encontraremos información acerca de las mareas y como se comportan en algún lugar en específico. Trabajaremos aspectos como el nivel del agua, las corrientes, las temperaturas, entre otros. 
 
Buscaremos cual es la relación que existe entre el sol y la luna (fuerzas gravitacionalas) que hace que se provoquen estos fenómenos. Para esto compararemos con algunas historia de datos. 

Trabajaremos con dos localizaciones escogidas en clase. En el siguiente informe nos enfocaremos en las ciudades de Los Angeles, CA. para el caso de Estados Unidos y Puerto Refugio, Baja California para datos mexicanos.
\newpage

Antes que nada es importante comenzar con los conceptos básicos como el que son las mareas, algúnas de sus características principals y entre otras cosas.  

\section{¿Qué son las mareas?}
\noindent Las mareas son los cambios periódicos de niveles de agua océano causada por efectos combinados de las fuerzas gravitacionales producidas por la luna y el sol y en otra parte por la rotación de la Tierra. 

Cabe destacar que esta fuerza interactúa en todo el planeta; tanto su parte solida como la líquida y gaseosa, pero en este trabajo nos enfocaremos en la parte líquida y más específicamente en los océanos y grandes lagos. 

Este fenómeno no sólo se limita a los océanos, también se presenta en otros sistemas en los cuales este presente la fuerza gravitacional y alcance un punto significativo de interacción con ese sistema.

Este fenómeno también puede ser causado por otros agentes, por ejemplo: lluvias, vientos, tornados, tormentas, tsunamis que también provocan variaciones del nivel del mar. 


\begin{center}
\includegraphics[scale=0.25]{mareas.jpg}
\end{center}

\newpage

 \section{Características de las mareas}

Los cambios de mareas provienen de las siguientes características:

\begin{itemize}
\item Nivel del mar se eleva durante varias horas, cubriendo la zona intermareal; pleamar.

\item Agua sube a su nivel más alto, alcanzando la marea alta.

\item Nivel del mar cae durante varias horas, revelando la zona intermareal; reflujo.

\item Agua deja de caer, alcanzando la marea baja.
\end{itemize}

 
 \section{Definiciones}
\noindent Existen algunas definiciones para las mareas dependiendo de su intensidad. A continuación se muestra un conjunto de definiciones que pueden describir a varios tipos de mareas.
\vspace{0.5cm}
 
 \includegraphics[scale=0.5]{2.png}
 
 \begin{itemize}
 

 \item {\textbf{Highest Astronomical Tide (HAT)}}
la marea más alta que se puede predecir que ocurra. Tenga en cuenta que las condiciones meteorológicas pueden agregar altura adicional al HAT.

\item{\textbf{Mean High Water Neaps (MHWN)}}
El promedio de las mareas altas en los días de primavera. 

\item{\textbf{Mean High Water Neaps (MHWN)}}
El promedio de las dos mareas altas en los días de marea muerta

\item{\textbf{Mean Sea Leve (MSL)}}
Este es el nivel medio del mar. El MSL es constante para cualquier localización durante un período largo.

\item{\textbf{Mean Low Water Neaps (MLWN)}}
Promedio de las dos mareas bajas en los días de mareas muertas.

\item{\textbf{Mean Low Water Springs (MLWS)}}
El promedio de las dos mareas bajas en los días de las mareas de primavera.

\item{\textbf{Lowest Astronomical Tide (LAT)}}
La marea más baja que se puede predecir que ocurra. 
 \end{itemize}
 
\section{Componentes de las mareas}
\noindent Los componentes primarios que constituyen a las mareas son la rotación de la tierra, la posición de la luna y el sol en relación con la tierra, la elevación de la luna sobre el ecuador y la batimetría. 

\section{Física}
\subsection{Historia}
\noindent El estudio de la física de mareas fue importante e el desarrollo temprano de heliocentrismo y la mecánica de los cuerpos celestes, con la existencia de dos mareas diarias explicadas por la gravedad de la luna. Tiempo después se estudió el efecto que provoca el Sol en las mareas.

El primero en conceptualizar la teoría de que las mareas eran provocadas por la luna fue Seleucus de Seleucia, alrededor del año 150 aC. Tiempo después científicos como Galileo Galilei, Newton, Kepler, etc. Continuaron estudiando las fuerzas que actúan sobre la tierra y repercuten en la producción de mareas.
\subsection{Fuerzas}
\noindent La fuerza de marea producida por un objeto masivo, tomando como referencia la luna como objeto de masa mínima necesaria, sobre una pequeña partícula situada sobre o en un cuerpo extenso (tierra en adelante) es la diferencia vectorial entre la fuerza gravitacional ejercida por la Luna sobre la partícula y la fuerza gravitacional que se ejercería sobre la partícula si estuviera situada en el centro de masa de la tierra. 
%%%%%%%%%%%%%%%%%%%%TABLA%%%%%%%%%%%%%%%%%%%%%%%%%%%%%%%
\section{Amplitud y ciclos de tiempo}
\begin{table}[htbp]
\begin{center}
\begin{tabular}{|l|l|l|}
\hline \hline
Nombre & Símbolo & Periodo (hrs)  \\
\hline \hline
Límite de agua superficial de la luna principal & $M_{4}$ & 6.210300601 \\ \hline
Límite de agua superficial de la luna principal & $M_{6}$ & 4.140200401 	\\ \hline
Agua superficial terdiurnal & $MK_{3}$ & 8.177140247 \\ \hline
Abundancia de agua poco profunda de la energía solar principal & $S_{3}$ & 6 \\ \hline
Cuarto de agua poco profunda diurna &  $MN_{4}$  & 6.269173724 \\ \hline
Principal lunar semidiurno & $M_{2}$ & 12.4206012 \\ \hline
Principal solar semidiurno & $S_{2}$ & 12 \\ \hline
Gran lunar elíptica semidiurna & $N_{2}$ & 12.65834751 \\ \hline
Lunar diurno & $K_{1}$ & 23.93447213  \\ \hline
Lunar diurno & $O_{1}$ & 25.81933871 \\ \hline
\end{tabular}
\label{tabla:sencilla}
\end{center}
\end{table}

\newpage
\section{Interpretación de datos}
\noindent En la siguiente sección analizaremos algunos datos mencionados en la introducción en los cuales se representa el comportamiento del nivel del mar y otras características. Para esto, como antes mencionamos, tomamos los datos de dos organizaciones encargadas de medir diariamente el comportamiento de algunas costas de México y de Estados Unidos. 

Graficaremos los datos obtenidos de tal manera que podamos observar la altura del nivel del agua en diferentes días.

\begin{itemize}
\item Los Angeles, CA:

En el caso de Estados Unidos analizaremos los datos de la organización NOAA,que nos dirá lo que sucedió en el mes de Enero del año 2016 en Los Angeles, CA. Cada dato fue tomado con una diferencia aproximada de 6 minutos.
\begin{center}
\includegraphics[scale=0.5]{LosAngeles.png}
\end{center}

\newpage
\item Ensenada, Baja California:

Para el caso de la medición por parte de la organización CICESE en Ensenada, Baja California Norte. En este caso, cada dato fue tomado con una diferencia aproximada de 1 hora y también corresponden a los datos de Enero de 2016
\begin{center}
\includegraphics[scale=0.5]{Ensenada.png}
\end{center}
\end{itemize}

Si comparamos las dos gráficas podemos encontrar bastantes similitudes, esto se debe a que estas localizaciones se encuentran relativamente cerca. En un principio se encuentran en la costa Oeste de Estados Unidos y México. 

\section{Conclusión}
\noindent A manera de conclusión podemos decir que las mareas pueden ser analizadas muy profundamente, que los procedimientos que fueron seguidos para realizar esta actividad pueden realizarse comunmente para analizar otro tipo de fenómenos.  
\newpage
\begin{thebibliography}{X}
\bibitem{Tides} \textsc{Wikipedia}
\textit{Tides}, Última edición: 21 de Febrero 2017, https://en.wikipedia.org/wiki/Tide a \today
\bibitem{Theory} \textsc{Wikipedia}
,
\textit{Teoría de las mareas}, Última edición: 17 de Enero de 2017, https://en.wikipedia.org/wiki/Theory\_of\_tides a \today


\bibitem{CICESE} \textsc{CICESE}
\textit{Calendario de mareas}, http://predmar.cicese.mx/calendarios/ a \today

\bibitem{NOAA} \textsc{NOAA}
\textit{Tides and currents}, https://tidesandcurrents.noaa.gov/waterlevels.html?id=9410660 a \today
\end{thebibliography}


\end{document}